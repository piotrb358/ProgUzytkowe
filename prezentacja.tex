
\documentclass[]{beamer}
%\usepackage[MeX]{polski}
%\usepackage[cp1250]{inputenc}
\usepackage{polski}
\usepackage[utf8]{inputenc}
\beamersetaveragebackground{blue!10}
\usetheme{Warsaw}
\usecolortheme[rgb={0.1,0.5,0.7}]{structure}
\usepackage{beamerthemesplit}
\usepackage{multirow}
\usepackage{multicol}
\usepackage{array}
\usepackage{graphicx}
\usepackage{enumerate}
\usepackage{amsmath} %pakiet matematyczny
\usepackage{amssymb} %pakiet dodatkowych symboli

\title{Nemegtomaia}
\date{}

\begin{document}

	\frame
{
\maketitle
}
\frame	
{
\frametitle{Pochodzenie}
Nemegtomaia – rodzaj wymarłego dinozaura z nadrodziny owiraptorozaurów, żyjącego na terenach dzisiejszej Mongolii w epoce kredy późnej, około 70 milionów lat temu. Pierwszy okaz znaleziono w 1996 i na jego podstawie w 2004 stworzono nowy rodzaj i gatunek N. barsboldi, pierwotnie z nazwą rodzajową Nemegtia, zmienioną na Nemegtomaia w 2005, gdyż poprzednia nazwa była już zajęta. Pierwsza część nazwy rodzajowej odnosi się do Kotliny Nemegt, gdzie znaleziono pozostałości zwierzęcia. Druga część oznacza „dobrą matkę”, stanowiąc odniesienie do faktu opieki nad jajami przez owiraptorozaury. Epitet gatunkowy honoruje mongolskiego paleontologa Rinczena Barsbolda. W 2007 znaleziono dwa kolejne okazy, z których jeden pozostał na szczycie gniazda wypełnionego jajami.
\begin{alertblock}
{coś}
tresc bloku
\end{alertblock}
}
	\frame
	{
\begin{figure}
\caption{Rekonstrukcja osobnika gniazdującego}
\centering
\includegraphics[width=0.7\textwidth]{Nesting_Nemegtomaia.jpg}
\end{figure}
}
	\frame
{
\begin{table}
\centering 
\begin{tabular}{|c|c|}
\hline
\multicolumn{2}{|c|}{Systematyka} \\
\hline
Domena&eukarionty \\
\hline
Królestwo&zwierzęta \\
\hline
Typ&strunowce \\
\hline
Podtyp&kręgowce \\
\hline
Gromada&zauropsydy \\
\hline
\end{tabular}
\end{table}
}
	\frame
{
x\left\{\begin{array}{l} $12x+3y+2z=0\\x+y-z=0\\7x+z=1$ \end{array}\in{R}
}
\end{document}
